\documentclass[11pt, oneside]{article}   	% use "amsart" instead of "article" for AMSLaTeX format
\usepackage[utf8]{inputenc}
\usepackage{geometry}                			% See geometry.pdf to learn the layout options. There are lots.
\geometry{letterpaper}                   			% ... or a4paper or a5paper or ... 
%\geometry{landscape}                			% Activate for rotated page geometry
%\usepackage[parfill]{parskip}    			% Activate to begin paragraphs with an empty line rather than an indent
\usepackage{graphicx}							% Use pdf, png, jpg, or eps§ with pdflatex; use eps in DVI mode
																% TeX will automatically convert eps --> pdf in pdflatex		
\usepackage{amssymb}
\usepackage{listings}
\usepackage{xcolor}
\usepackage[export]{adjustbox}
\usepackage{booktabs}
\usepackage{float}
\usepackage{tikz}
\usepackage{pgfplots}
\usepgfplotslibrary{fillbetween}
\usetikzlibrary{patterns}
\pgfplotsset{compat=1.16}

\definecolor{codegreen}{rgb}{0,0.6,0}
\definecolor{codegray}{rgb}{0.5,0.5,0.5}
\definecolor{codepurple}{rgb}{0.58,0,0.82}
\definecolor{backcolour}{rgb}{0.95,0.95,0.92}

\lstdefinestyle{mystyle}{
	backgroundcolor=\color{backcolour},   
	commentstyle=\color{codegreen},
	keywordstyle=\color{magenta},
	numberstyle=\tiny\color{codegray},
	stringstyle=\color{codepurple},
	basicstyle=\ttfamily\footnotesize,
	breakatwhitespace=false,         
	breaklines=true,                 
	captionpos=b,                    
	keepspaces=true,                 
	numbers=left,                    
	numbersep=5pt,                  
	showspaces=false,                
	showstringspaces=false,
	showtabs=false,                  
	tabsize=2
}

\lstset{style=mystyle}

\title{Functional Testing}
\author{Zelin Cai, Patrick Silvestre}
\date{}

\begin{document}
\maketitle

\section{Input Domain}
The input domain of the unit \texttt{nextDate} consists of dates formatted as follows: $DD/MM/YYYY$. For months, $MM$, the input condition specifies the range $[1, 12]$. If $MM = 2$ and the year is not a leap year, then the input condition range for dates, $DD$ is $[1, 28]$ (a year is a leap year if divisible by 4 and not divisible by 100 (unless divisible by 400)). If $MM = 2$  and the year is a leap year, then the input condition range for $DD$ is $[1, 29]$. If $MM = 4, 6, 9,$ or $11$, then the input condition range for $DD$ is $[1, 30]$. If $MM = 1, 3, 5, 7, 8, 10,$ or $12$, then the input condition range for $DD$ is $[1, 31]$.

\section{Equivalence Classes}
In order to identify equivalence classes (ECs), the following strategy was used: input conditions typically specified a range $[a, b]$, thus one EC valid input was identified for $a \leq X \leq b$, and two others with invalid input were identified for $X < a$ and $b < X$.

\subsection{Months}
	\begin{itemize}
		\item{EC-01: $01 \leq MM \leq 12$}
		\item{EC-02: $MM < 01$}
		\item{EC-03: $12 < MM$}
	\end{itemize}

\subsection{Dates}
\subsubsection{MM = 2, YY is not a leap year}
	\begin{itemize}
		\item{EC-04: $01 \leq DD \leq 28$}
		\item{EC-05: $DD < 01$}
		\item{EC-06: $28 < DD$}
	\end{itemize}

\subsubsection{MM = 2, YY is a leap year}
	\begin{itemize}
		\item{EC-07: $01 \leq DD \leq 29$}
		\item{EC-08: $DD < 01$}
		\item{EC-09: $29 < DD$}
	\end{itemize}	
	
\subsubsection{MM = 4, 6, 9, or 11}
	\begin{itemize}
		\item{EC-10: $01 \leq DD \leq 30$}
		\item{EC-11: $DD < 01$}
		\item{EC-12: $30 < DD$}
	\end{itemize}
	
\subsubsection{MM = 1, 3, 5, 7, 8, 10, or 12}
	\begin{itemize}
		\item{EC-13: $01 \leq DD \leq 31$}
		\item{EC-14: $DD < 01$}
		\item{EC-15: $31 < DD$}
	\end{itemize}

\subsection{Years}
	\begin{itemize}
		\item{EC-16: $1900 \leq YYYY \leq 2099$}
		\item{EC-17: $YYYY < 1900$}
		\item{EC-18: $2099 < YYYY$}
	\end{itemize}

\newpage
\section{Test Cases}
\subsection{Test Cases from Equivalence Classes}
In order to identify test cases (TCs) from ECs, the following strategy is used: For each EC with valid input that has not been covered by a TC, write a TC covering as many uncovered ECs as possible. Then, for each EC with invalid input that has been covered, write a new TC that covers only that EC.

\subsubsection{Test Cases for ECs with Valid Input}
Valid Input ECs: EC-01, EC-04, EC-07, EC-10, EC-13, EC-16

\begin{table}[!htb]
\centering
\begin{tabular}{|l|l|l|l|l|l|l|l|}
\hline
TC \# & TC         & EC-01      & EC-04      & EC-07      & EC-10      & EC-13      & EC-16      \\ \hline
01   & 01/02/1999 & \checkmark & \checkmark &            &            &            & \checkmark \\ \hline
02   & 01/02/2000 & \checkmark &            & \checkmark &            &            & \checkmark \\ \hline
03   & 01/04/2000 & \checkmark &            &            & \checkmark &            & \checkmark \\ \hline
04   & 01/01/2000 & \checkmark &            &            &            & \checkmark & \checkmark \\ \hline
\end{tabular}
\end{table}
	
\subsubsection{Test Cases for ECs with Invalid Input}
Invalid Input ECs: EC-02, EC-03, EC-05, EC-06, EC-08, EC-09, EC-11, EC-12, EC-14 EC-15, EC-17, EC-18

\begin{table}[!htb]
\centering
\begin{tabular}{|l|l|l|l|l|l|l|l|}
\hline
TC \# & TC         & EC-02      & EC-03      & EC-05      & EC-06      & EC-08      & EC-09      \\ \hline
05    & 01/00/2000 & \checkmark &            &            &            &            &            \\ \hline
06    & 01/13/2000 &            & \checkmark &            &            &            &            \\ \hline
07    & 00/02/1999 &            &            & \checkmark &            &            &            \\ \hline
08    & 29/02/1999 &            &            &            & \checkmark &            &            \\ \hline
09    & 00/02/2000 &            &            &            &            & \checkmark &            \\ \hline
10    & 30/02/2000 &            &            &            &            &            & \checkmark \\ \hline
\end{tabular}
\end{table}

\begin{table}[!htb]
\centering
\begin{tabular}{|l|l|l|l|l|l|l|l|}
\hline
TC \# & TC         & EC-11      & EC-12      & EC-14      & EC-15      & EC-17      & EC-18      \\ \hline
11    & 00/04/2000 & \checkmark &            &            &            &            &            \\ \hline
12    & 31/04/2000 &            & \checkmark &            &            &            &            \\ \hline
13    & 00/01/2000 &            &            & \checkmark &            &            &            \\ \hline
14    & 32/01/2000 &            &            &            & \checkmark &            &            \\ \hline
15    & 01/01/1899 &            &            &            &            & \checkmark &            \\ \hline
16    & 01/01/2100 &            &            &            &            &            & \checkmark \\ \hline
\end{tabular}
\end{table}

\subsection{Test Cases for Boundary Value Analysis}
\subsection{Test Cases for Random Testing}

\end{document} 